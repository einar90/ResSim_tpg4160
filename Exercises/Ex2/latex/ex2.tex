\documentclass[english,11pt,a4paper]{article}
\usepackage[T1]{fontenc} % --------------| More characters.
\usepackage[utf8]{inputenc} % -----------| Direct use of scandinavian letters.
\usepackage{float} % --------------------| More options for floats.
\usepackage{graphicx} % -----------------| Support more image formats.
\usepackage{booktabs} % -----------------| Better-looking tables.
\usepackage{tabularx} % -----------------| Better tables
\usepackage{subcaption} % ---------------| Subfigures.
\usepackage[a4paper]{geometry} % --------| Adjusting page margins.
\usepackage{amsmath,amssymb,amsfonts} % -| Various math, including eqref.
\usepackage{xcolor} % -------------------| Allows defn. of custom colors.
\usepackage{babel}
\usepackage{url}
\usepackage{enumitem}

% XY-pic. Used for creating illustrations.
\input xy
\xyoption{all}

% Styling captions.
\usepackage{caption}
\captionsetup{margin=10pt,font=small,labelfont=bf}


%******************************************************************************
% Includes the listings package and sets some settings for it.
% REQUIRES the `color' package.
%******************************************************************************

\usepackage{listings} % -----------------------| Used for source code listings.
\definecolor{lst-gray}{RGB}{100,100,100}  % ---|Color for line-numbers.
\definecolor{lst-light-gray}{RGB}{250,250,250} % Background color for listings.
\lstset{
  aboveskip=0em, % -------------------| Skip above listing box.
  backgroundcolor=\color{lst-light-gray}, % Background color.
  basicstyle=\ttfamily\scriptsize, % -| Default font style.
  belowskip=\topskip, % --------------| Skip below listing box.
  breakatwhitespace=false, % ---------| Automatic breaks only at whitespace?.
  breaklines=true, % -----------------| Sets automatic line breaking.
  captionpos=t, % --------------------| Sets the caption-position to bottom.
  commentstyle=\color{green}, % ------| Comment style.
  escapeinside={\%*}{*)}, % ----------| For adding LaTeX within code.
  frame=single, % --------------------| Adds a frame around the code.
  keepspaces=true, % -----------------| Keeps spaces in text.
  keywordstyle=\color{blue}, %--------| Keyword style.
  language={C}, % --------------------| The language of the code.
  literate={æ}{{\ae}}1 % -------------| Character conversions
           {Æ}{{\AE}}1
           {ø}{{\oe}}1
           {Ø}{{\OE}}1
           {å}{{\aa}}1
           {Å}{{\AA}}1
           {µ}{{\ensuremath{\mu}}}1,
  numbers=left, % --------------------| Line-number position: none/left/right.
  numbersep=5pt, % -------------------| Distance between line-numbers and code.
  numberstyle=\tiny\color{lst-gray}, %| The style that used for line-numbers.
  rulecolor=\color{black}, % ---------| Frame color.
  showspaces=false, % ----------------| Show spaces with underscores.
  showstringspaces=false, % ----------| Underline spaces within strings.
  showtabs=false, % ------------------| Show tabs with underscores.
  stepnumber=2, %---------------------| Step between two line-numbers..
  stringstyle=\color{red}, % ---------| String literal style.
  tabsize=4, % -----------------------| Sets default tabsize to 2 spaces.
  title=\lstname % -------------------| Show the filename of included file.
}
\input{xy}

\begin{document}

\title{Reservoir Simulation, Exercise 2}
\author{Einar Baumann}
\maketitle
\thispagestyle{empty}


\section*{b) Varying number of grid blocks} % (fold)
\label{sec:b_varying_number_of_grid_blocks}
The plot for this case is shown in Figure~\ref{fig:caseb}. We see that increasing the number of grid blocks results in a curve ever closer to piston displacement, which is what we would expect from an ``exact'' solution here because we have no capillary pressure.
% section b_varying_number_of_grid_blocks (end)


\section*{c) Varying time step size} % (fold)
\label{sec:c_varying_time_step_size}
The plot for this case is shown in Figure~\ref{fig:casec}. We don't see a lot of change here when we vary the size of the time step, but increasing does give a slightly sharper displacement front for some reason.
% section c_varying_time_step_size (end)


\section*{d) Varying mobilities} % (fold)
\label{sec:d_varying_mobilities}
The plot for this case is shown in Figure~\ref{fig:cased}. Decreasing the weighting factors for the mobilities moves the curve further away from the ``ideal'' curve, through something that could be expected for a calculation made with \emph{wighted average selection}, and on to something that's just wrong.
% section d_varying_mobilities (end)


\section*{e) Varying capillary pressure multiplier} % (fold)
\label{sec:e_varying_capillary_pressure_multiplier}
The plot for this case is shown in Figure~\ref{fig:casee}. We that that increasing the capillary pressure multiplier makes the model deviate more and more from piston displacement, which is what we expect when the capillary pressure increases.
% section e_varying_capillary_pressure_multiplier (end)



\begin{figure}[H]
  \centering
  \includegraphics[width=0.83\textwidth]{../code/plotting/b.pdf}
  \caption{Plot for case b).}
  \label{fig:caseb}
\end{figure}

\begin{figure}[H]
  \centering
  \includegraphics[width=0.83\textwidth]{../code/plotting/c.pdf}
  \caption{Plot for case c).}
  \label{fig:casec}
\end{figure}

\begin{figure}[H]
  \centering
  \includegraphics[width=0.83\textwidth]{../code/plotting/d.pdf}
  \caption{Plot for case d).}
  \label{fig:cased}
\end{figure}

\begin{figure}[H]
  \centering
  \includegraphics[width=0.83\textwidth]{../code/plotting/e.pdf}
  \caption{Plot for case e).}
  \label{fig:casee}
\end{figure}



\clearpage
\section{Program listings} % (fold)
\label{sec:program_listings}

\lstinputlisting[%
  caption={plot.py -- Used to generate the plots},
  label={lst:plot},
  language={Python}]
  {../code/plotting/plot.py}

% section program_listings (end)

\end{document}