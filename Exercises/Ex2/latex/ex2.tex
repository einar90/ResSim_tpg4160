\input{../../common/preamble.tex}
\input{../../common/preamble-addon-listings.tex}
\input{xy}

\begin{document}

\title{Reservoir Simulation, Exercise 2}
\author{Einar Baumann}
\maketitle
\thispagestyle{empty}


\section*{b) Varying number of grid blocks} % (fold)
\label{sec:b_varying_number_of_grid_blocks}
The plot for this case is shown in Figure~\ref{fig:caseb}. We see that increasing the number of grid blocks results in a curve ever closer to piston displacement, which is what we would expect from an ``exact'' solution here because we have no capillary pressure.
% section b_varying_number_of_grid_blocks (end)


\section*{c) Varying time step size} % (fold)
\label{sec:c_varying_time_step_size}
The plot for this case is shown in Figure~\ref{fig:casec}. We don't see a lot of change here when we vary the size of the time step, but increasing does give a slightly sharper displacement front for some reason.
% section c_varying_time_step_size (end)


\section*{d) Varying mobilities} % (fold)
\label{sec:d_varying_mobilities}
The plot for this case is shown in Figure~\ref{fig:cased}. Decreasing the weighting factors for the mobilities moves the curve further away from the ``ideal'' curve, through something that could be expected for a calculation made with \emph{wighted average selection}, and on to something that's just wrong.
% section d_varying_mobilities (end)


\section*{e) Varying capillary pressure multiplier} % (fold)
\label{sec:e_varying_capillary_pressure_multiplier}
The plot for this case is shown in Figure~\ref{fig:casee}. We that that increasing the capillary pressure multiplier makes the model deviate more and more from piston displacement, which is what we expect when the capillary pressure increases.
% section e_varying_capillary_pressure_multiplier (end)



\begin{figure}[H]
  \centering
  \includegraphics[width=0.83\textwidth]{../code/plotting/b.pdf}
  \caption{Plot for case b).}
  \label{fig:caseb}
\end{figure}

\begin{figure}[H]
  \centering
  \includegraphics[width=0.83\textwidth]{../code/plotting/c.pdf}
  \caption{Plot for case c).}
  \label{fig:casec}
\end{figure}

\begin{figure}[H]
  \centering
  \includegraphics[width=0.83\textwidth]{../code/plotting/d.pdf}
  \caption{Plot for case d).}
  \label{fig:cased}
\end{figure}

\begin{figure}[H]
  \centering
  \includegraphics[width=0.83\textwidth]{../code/plotting/e.pdf}
  \caption{Plot for case e).}
  \label{fig:casee}
\end{figure}



\clearpage
\section{Program listings} % (fold)
\label{sec:program_listings}

\lstinputlisting[%
  caption={plot.py -- Used to generate the plots},
  label={lst:plot},
  language={Python}]
  {../code/plotting/plot.py}

% section program_listings (end)

\end{document}