% -*- root: ../ressim.tex -*-

\section{Band structure and bandwidth of equation systems} % (fold)
\label{sec:band_structure_and_bandwidth_of_equation_systems}

\begin{question}
  Sketch the coefficient matrix for the following systems, indicating non-zero diagonals with
approximate lines. Label the diagonals. What is the bandwidth?
\end{question}

The bandwidth may be computed as
\begin{equation}
  N_b = 2N_{[x,y]}+1
\end{equation}
depending on the equation numbering scheme.

\subsection{One-dimensional, one phase flow} % (fold)
\label{sub:one_dimensional_}

\begin{question}
  Sketch the coefficient matrix for one-dimensional, one phase flow, with the pressure equation
  \begin{equation}
    \nonumber
    a_i P_{i-1} + b_i P_i + c_i P_{i+1} = d_i, \; i=1,N
  \end{equation}
  applicable to the following system:

  \begin{center}
    \includegraphics[]{illustrations/diffusivity/1d-system-discrete.pdf}
  \end{center}
\end{question}

\begin{equation}
  N_b = 3 \nonumber
\end{equation}

\begin{center}
  \includegraphics[]{illustrations/matrix-structures/1d-1phase.pdf}
\end{center}
% subsection one_dimensional_ (end)


\subsection{Two-dimensional, one phase flow} % (fold)
\label{sub:two_dimensional_one_phase_flow}

\begin{question}
  Sketch the coefficient matrix for two-dimensional, one phase flow with the pressure equation:
  \begin{equation}
    \nonumber
    e_{i,j}P_{i,j-1}+a_{i,j}P_{i-1,j}+b_{i,j}P_{i,j}+c_{i,j}P_{i+1,j}+f_{i,j}P_{i,j+1}=d_{i,j},\; i=1,N_{x},\; j=1,N_{y}
  \end{equation}
  applicable to the following grid system:

  \begin{center}
    \includegraphics[]{illustrations/matrix-structures/2d-system-numbering.pdf}
  \end{center}
\end{question}

\begin{equation}
  N_b = 2N_x + 1 = 2\cdot 6 + 1 = 13 \nonumber
\end{equation}

\begin{center}
  \includegraphics[]{illustrations/matrix-structures/2d-2phase.pdf}
\end{center}

% subsection two_dimensional_one_phase_flow (end)

\subsection{Two-dimensional, two phase flow, alternative numbering} % (fold)
\label{sub:two_dimensional_two_phase_flow_alternative_numbering}

\begin{question}
  As above, but now the numbering of the grid starts in the $j$-direction:

  \begin{center}
    \includegraphics[]{illustrations/matrix-structures/2d-system-numbering-alt.pdf}
  \end{center}
\end{question}

\begin{equation}
  N_b = 2N_y+1 = 2\cdot 8 + 1 = 17 \nonumber
\end{equation}

\begin{center}
  \includegraphics[]{illustrations/matrix-structures/2d-2phase-alt.pdf}
\end{center}

% subsection two_dimensional_two_phase_flow_alternative_numbering (end)

\subsection{Three-dimensional, one phase flow} % (fold)
\label{sub:three_dimensional_one_phase_flow}

\begin{question}
  Three dimensional, one phase flow, with the pressure equation:
  \begin{multline*}
    g_{i,j,k}P_{i,j,k-1}+e_{i,j,k}P_{i,j-1,k}+a_{i,j,k}P_{i-1,j,k}+b_{i,j,k}P_{i,j,k} \\
    +c_{i,j,k}P_{i+1,j,k}+f_{i,j,k}P_{i,j+1,k}+h_{i,j,k}P_{i,j,k+1}=d_{i,j,k}, \\
    i=1,N_{x},j=1,N_{y},k=1,N_{z}
  \end{multline*}
  applicable to the following grid system (grid blocks numbered in the sequence of $x,y,z$):

  \begin{center}
    \includegraphics[scale=0.75]{illustrations/matrix-structures/3d-system-numbering.pdf}
  \end{center}
\end{question}

Bandwidth, when numbering along first $x$, then  $y$, then $z$:
\begin{equation}
  N_b = 2N_x N_y +1 \nonumber
\end{equation}

\begin{center}
  \includegraphics[]{illustrations/matrix-structures/3d-1phase.pdf}
\end{center}
% subsection three_dimensional_one_phase_flow (end)


\subsection{Solving 1D, one phase flow system} % (fold)
\label{sub:solving_1d_one_phase_flow_system}

\begin{question}
  For the 1D system above, outline how it is solved for pressures using the Gaussian elimination method.
\end{question}
A $9\times 9$ matrix is here shown as an example:
\small{%
\[
  \begin{bmatrix}
    b & c & 0 & 0 & 0 & 0 & 0 & 0 & 0 \\
    a & b & c & 0 & 0 & 0 & 0 & 0 & 0 \\
    0 & a & b & c & 0 & 0 & 0 & 0 & 0 \\
    0 & 0 & a & b & c & 0 & 0 & 0 & 0 \\
    0 & 0 & 0 & a & b & c & 0 & 0 & 0 \\
    0 & 0 & 0 & 0 & a & b & c & 0 & 0 \\
    0 & 0 & 0 & 0 & 0 & a & b & c & 0 \\
    0 & 0 & 0 & 0 & 0 & 0 & a & b & c \\
    0 & 0 & 0 & 0 & 0 & 0 & 0 & a & b \\
  \end{bmatrix}
\]
}

We see that our matrix only consists of 3 non-zero diagonal, thus our Gaussian elimination algorithm may be simplified to operate only on the band itself, and ignore the rest of the matrix.

We begin by foreward eliminating the lower diagonal ($a$):
\[
  b_i = b_i - c_{i-1} (a_i / b_{i-1}), \quad i=2,N
\]
\[
  d_i = d_i - d_{i-1} (a_i / b_{i-1}), \quad i=2,N
\]
Then the last pressure is solved for:
\[
  P_N = d_N / b_N
\]
And finally the remaining pressures are solved by backward substitution:
\[
  P_i = d_i - c_i P_{i+1}/b_i, \quad i = N-1, 1
\]

% subsection solving_1d_one_phase_flow_system (end)

% section band_structure_and_bandwidth_of_equation_systems (end)
